% !TEX TS-program = luatex
\documentclass{cv}
\name{RYAN}{WHITELL}
\tagline{Software Engineer}

\socialinfo{
	\address{Seattle, Washington}
	\smartphone{(206) 512-5204}
	\email{whitellryan@gmail.com}
	\linkedin{ryanwhitell}
}

\begin{document}
	\makecvheader

	\par{
		I am a Software Engineer with a strong focus on MLOps. My Master's thesis was an exploration of machine learning, particularly deep learning, for music recommendation. My primary goal is to build robust MLOps systems that enable the scalability, reliability, and optimal performance of production ML models, as well as accelerate the process of scientists getting their ideas into production. 
		
		I am happiest working on music or gaming-related products.

		Seasoned Software Engineer with 5+ years of industry experience across diverse domains. I started my career developing a large-scale 3D .NET application for naval architects using C++ and SQL. I then transitioned into full-stack web development, working with React, GQL, and Java-based backend services for Amazon's Prime Gaming website. In my most recent role, I have specialized in MLOps and Data Engineering for Amazon's Game Growth organization, empowering scientists to better answer game industry related questions.
		
		I am seeking a new opportunity in the games industry to apply my skills to a product I love and to learn new industry relevant skills.
	}

	\bigskip
	
	\sectionTitle{Experience}
	\begin{experiences_env}
		\experienceWithProjects
		   {June 2023}      {Amazon}{Machine Learning SDE II}
			{Jan 2022}      {
								Designed and built machine learning infrastructure for gaming related science endeavors at scale. Provided ad-hoc software support to scientists.
							}
							{Python, SQL, Redshift, SageMaker, Docker, Airflow, MLOps}
							{
								\begin{itemize}
									\item \textbf{\textsc{LTV Model:}} Deployed a model at scale to predict customer lifetime value on a monthly basis for millions of customers. Collaborated with data scientists to convert the model into a production-ready ML system using MLOps principles, and managed the pipeline using AWS services.
									\item \textbf{\textsc{IGDB Partnership:}} Initiated a partnership between Prime Gaming and the Internet Game Database (IGDB) team to share game metadata. Drove alignment with leadership to adopt the IGDB game ID as the primary game ID in the Prime Gaming catalog. By doing so, Prime Gaming gained a rich set of game metadata to join its catalog items to for improved analytics. Independently built the IGDB data ingestion pipeline.
								\end{itemize}
							}
		\emptySeparator
		\experienceWithProjects
       {December 2021}      {Amazon}{SDE II}
			{May 2019}      {
								Designed, implemented, and deployed software components and features for the Prime Gaming website; identified and solved problems, improved team’s software and development/testing processes, gparticipated in hiring and mentoring, led cross-team projects (with a focus on analytics, telemetry, and experimentation). Advised on recommendation systems for Prime Gaming.
								
								Designed, implemented, and deployed software components for Prime Gaming, improved team processes, and led cross-team projects with a focus on analytics, telemetry, and experimentation.
							}
							{Java, JavaScript, React, GraphQL, DynamoDB, Kinesis}
							{
								\begin{itemize}
									\item \textbf{\textsc{Fiber:}} Redesigned a legacy metrics reporting system and led a team of SDEs and DEs to complete the work. Achieved a 100\% reduction in development hours for customer event metric changes by using a semi-structured data model. This laid the groundwork for broader analytics efforts at Prime Gaming by solving architectural bottlenecks that had been stifling innovation.
									\item \textbf{\textsc{Recommendation at Prime Gaming:}} Designed a high-level overview of the technical infrastructure needed to power production ready, customer facing machine learning models at Prime Gaming. Cleared up technical ambiguities and made recommendations to leaders regarding existing architecture integrations, timelines, and new team compositions. This work has helped pushed the idea through funding and into pre-development.
									\item \textbf{\textsc{Pixel Management:}} Designed and implemented a system for content managers to schedule tracking pixels on gaming.amazon.com. Self-service pixel management empowered the performance marketing team to utilize intelligent re-targeting methods to build personalized ads for customers. This improved Return on Ad Spend (RoAS) and saved millions in advertising costs per year.
								\end{itemize}
							}
		\emptySeparator
		\experienceWithoutProjects
			{January 2019}  {ShipConstructor}{Software Developer}
			{June 2017}     {
								Developed a database-driven Autodesk shipbuilding and 3D modeling app for naval architects and marine engineers. Worked on the data team using C/C++, C\# (.NET), and T-SQL (Microsoft SQL Server) in an agile environment. Contributed to client database performance improvement efforts.
							}
							{SQL, C, C++, C\#, .NET, Microsoft SQL Server}
		\emptySeparator
	\end{experiences_env}

	\sectionTitle{Education}
	\begin{education_env}
		\education
		{Regis University}{2017 - 2019}
		{Master of Science (M.S.), Software Engineering}{4.0}
		{\website{https://scholar.google.com/citations?view_op=view_citation&hl=en&user=Grtp_AsAAAAJ&citation_for_view=Grtp_AsAAAAJ:QIV2ME_5wuYC}{Thesis: Content-Based Music Recommendation using Deep Learning}}
					
		\education
		{Regis University}{2015 - 2017}
		{Bachelor of Science (B.S.), Computer Science}{3.99}
		{Completed the degree online while playing professional hockey.}

		\education
		{Norwich University}{2013 - 2015}
		{Bachelor of Science (B.S.), Electrical and Computer Engineering}{3.95}
		{Note: Transferred into Regis University with one academic year remaining and did not obtain the degree.}
	\end{education_env}

	\makecvfooter
	{}
	{\textsc{Ryan Whitell - Resume}}
	{}

\end{document}